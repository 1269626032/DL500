\chapter{第十九章
软件(算法)专利申请及权利保护}\label{ux7b2cux5341ux4e5dux7ae0-ux8f6fux4ef6ux7b97ux6cd5ux4e13ux5229ux7533ux8bf7ux53caux6743ux5229ux4fddux62a4}

% \begin{verbatim}
% Markdown Revision 1;
% Date: 2019/07/16
% Editor: 何建宏
% Contact: bonopengate@gmail.com
% \end{verbatim}

\section{19.1
为什么需要对软件(算法)进行保护?}\label{ux4e3aux4ec0ux4e48ux9700ux8981ux5bf9ux8f6fux4ef6ux7b97ux6cd5ux8fdbux884cux4fddux62a4}

  对软件/系统/算法进行保护可以有效地保护在计算机领域中的公司或个人的权益,随着人工智能的兴起,在图像处理、语音处理、文本处理等方向上,公司或个人不断地研发新的系统,探究新的算法,可是随着软件的开发/设计成本逐渐增高,愈来愈多公司或个人开始对对手产品进行模仿,而这个过程中,被模仿的公司或个人也是深受其害。在美国早已有专门设立的对软件的专利保护政策,而国内因为各种因素迟迟未有这方面的实行政策。所以目前软件开发者或者算法设计者只能通过其它的方法来保护自己的权益,保护自己的知识产权。【1】

  近期谷歌对 Dropout 算法的申请成功,以及在路上的对 Word2Vec
算法的申请掀起了一阵对专利的不解之潮以及对商业公司的这种专利申请行为的恐惧之潮,仿佛自己会在不知不觉中踏入了别人的专利的保护范围,从而让自己遭受到利益损伤;事实上,这种危机感是对的,目前国外的科技公司都在积极地对自己的算法或者系统进行专利申请,而这些申请迟早也会危及到需要在国外开展商业活动的公司或者个人。

  谈到知识产权,让我们把视线移回到国内,国内目前不管是阿里、百度、腾讯亦或是字节跳动、商汤等公司,都已经开始为自己公司的知识产权进行布局:在各种专利申请网站上,每个月都会有几十乃至上百个算法相关专利在申请,而他们的直接产品------软件/系统也是他们当前要进行专利申请以及知产保护的主战场,开战已久。

  其实,对自己研发的软件/系统进行恰当的保护除了可以有效地降低自己在未来遭遇诉讼的风险,也让自己可以在未来的``被侵权''中可以提供更多的证据来有效夺回自己的利益,目前在华为也已经出现了售卖专利使用权的方式来获取远超开发乃至申请专利所消耗的资源的盈利------利用每年几万件的专利申请,将自己的知识产权和自己的开发成果牢牢保护在自己的专利墙中。

  在当下,腾讯等国内领先的科技公司都不再停留在简单的软著保护,而是开始对软件的执行方法乃至算法进行保护,从而将自己的技术紧紧保护在自己的专利墙中。在法律愈来愈完善的情况下,专利的申请与保护也愈来愈重要。

\subsection{19.1.1
如果不保护自己的软件(算法)会怎样?}\label{ux5982ux679cux4e0dux4fddux62a4ux81eaux5df1ux7684ux8f6fux4ef6ux7b97ux6cd5ux4f1aux600eux6837}

  在《中国法院知识产权司法保护状况(2018)》中指出,2018年,人民法院共新收一审、二审、申请再审等各类知识产权案件334951件,审结319651件(含旧存),比2017年分别上升41.19\%和41.64\%。其中,竞争类一审案件数量
(含垄断民事案件)增幅最为显著,同比上升63.04\%,达到4146件。其中的新收专利案件为21699,同比上升35.53\%。而其中有一个特别的就是百度诉搜狗侵权的案例,判决书为\textbf{北京市高级人民法院(2018)京民终498号民事判决书},诉讼的内容就是输入法的操作本身,这个和申诉算法或者软件相关专利其实已经十分地类似了,如果搜狗败诉,赔偿额则是一千万,但是因为搜狗本身拥有自己的输入法操作的发明专利,因此百度与它的专利纠纷最后以百度败诉收场。

  目前因为算法专利而纠纷的案件在国内尚少,但是其在美国,\textbf{RSA}可以就是凭着自己的算法专利活着的;因为目前基于算法的技术方案往往会大于算法本身,如果申请成功,可以很容易针对别人的应用进行诉讼------而这个别人不一定不会是自己。

\section{19.2
软件或者系统该申请怎样的保护?}\label{ux8f6fux4ef6ux6216ux8005ux7cfbux7edfux8be5ux7533ux8bf7ux600eux6837ux7684ux4fddux62a4}

  仅指中国范围内(包括台湾以及香港):一般来说,有对方法以及使用步骤进行保护的\textbf{软件专利}以及对代码进行保护的\textbf{软件著作权},但是少数情况下,也可以对自己的软件界面进行\textbf{外观专利}的保护。

  如果你目前身处美国,可以直接对算法进行\textbf{算法专利}申请。

\section{19.3
如何保护自己的软件或算法?}\label{ux5982ux4f55ux4fddux62a4ux81eaux5df1ux7684ux8f6fux4ef6ux6216ux7b97ux6cd5}

  论文中出现的算法无法直接申请专利,例如无法对E=mc\^{}2进行专利申请。\textbf{但是可以就某个或者某些领域对算法进行申请(例如根据这个而制作出来的氢弹或者原子弹),此处与其它的方法类的发明撰写技巧比较不一样。}

  以下解释为什么无法直接对算法(非应用至特定领域)申请专利:

  根据专利法第二条的规定,专利法的保护客体为

  ①对产品、方法或者其改进所提出的新的技术方案\textbf{(发明)}

  ②对产品的形状、构造或者其结合所提出的适于实用的新的技术方案\textbf{(实用新型)}

根据专利法第二十五条的规定,智力活动的规则和方法不属于可授予专利权的客体范围。【2】

  \textbf{智力活动},是指人的思维运动,它源于人的思维,经过推理、分析和判断产生出抽象的结果,或者必须经过人的思维运动作为媒介,间接地作用于自然产生结果。智力活动的规则和方法是指导人们进行思维、表述、判断和记忆的规则和方法。\textbf{智力活动在计算机领域常指的就是算法,例如红黑树算法、蚁群算法等;或者在论文中常出现的数学公式。}

  由于其没有采用技术手段或者利用自然规律(因为智力活动是属于个人的思考流程),也未解决技术问题和产生技术效果,因而不构成技术方案。因而它既不符合\textbf{专利法第二条第一以及第二款的规定},
又属于\textbf{专利法第二十五条规定的情形}。因此,指导人们进行这类活动的规则和方法不能被授予专利权。

\subsection{19.3.1 申请例子①}\label{ux7533ux8bf7ux4f8bux5b50}

  上文中也提到了可以通过就某个领域或者某些领域对该算法本身进行保护。

  专利公告号为\textbf{CN105095257B}的\textbf{一种用户信用度确定方法、装置和媒体信息推荐系统},申请人为腾讯科技(北京)有限公司

  其中的独权一展示的就是利用现有的算法对具体的被定义的实体:贡献因子(推荐系统所用的参数)、信用度(推荐系统所用的参数)以及媒体信息(文章内容),得到定义的实体------推荐值(推荐系统所用的参数)

  此处可以学习的地方有以下几个:

\begin{itemize}
% \tightlist
\item
  使用文书语言对自己的算法进行高度概括,贡献因子和信用度如何被使用,媒体信息如何被使用,以及最后得出的推荐值如何被使用等。
\item
  专利的撰写格式,专利有固定的撰写格式,而这个主要是为了可以在后续的审查中不会因为撰写格式的问题导致下非二十六条第二款以外的审查意见或者补正通知。
\item
  专利的行文规范以及布局,专利的布局是为了彻底保护自己的专利(此处主要是你用这个算法实现的方案),行文规范则是可以让你在后续发生专利纠纷的时候不会因为行文问题,导致自己的专利会被无效或者轻易地进行规避。
\end{itemize}

  上述这个是第一条独权,保护的范围为用户的信用度的确定方法。可是如果仅仅只有这个方法的话,其实是很难在审查中通过审查的,因为审查员可以简单地就第二十五条对你进行辩驳。\textbf{但是这个是合法的申请},因为你这个是将推荐系统的算法用于了媒体信息的推荐中,所以就脱离了智力活动的范畴,而是形成了一个\textbf{对产品、方法或者其改进所提出的新的技术方案}。其中媒体信息是上位概念,可以包括但不限制于文字、视频、音频等,保护效果会更大,也可以有效地限制别人使用相同或者相似的方式进行推荐系统的搭建。

  可是如果仅仅是申请了这个方法,那还是不够的,你需要对这个算法本身的数据采录以及数据保存、以及数据处理进行保护的话,你需要保护的还有对执行这个算法的本身的装置(使用过程)、系统(使用场景)等。所以在上述的这一份专利中,会出现了四条独立主权,涵盖了主题的三大部分:

\begin{itemize}
% \tightlist
\item
  用户信用度确定方法
\item
  装置
\item
  媒体信息推荐系统
\end{itemize}

  而这四条主要为了进一步保护你的算法以及使用这类算法的场景。从而构建起``专利墙''

\subsection{19.3.2 申请例子②}\label{ux7533ux8bf7ux4f8bux5b50-1}

  专利公开号为\textbf{CN110009090A}的\textbf{神经网络训练与图像处理方法及装置},申请人为北京市商汤科技开发有限公司。(在写当前文章时,该专利还处于审查当中(该例子比较特殊,因为应用场景上位至图像领域))

  我们如果下载了专利文本进行检查的话,会发现它也是将自己的方法分成了多个独立主权,虽说实际上保护的是同一样流程(算法),但是必须声明存在装置使用了上述的算法以及存在内存存储了上述的算法。

  根据独权一,我们可以得到三个被实体化的参数:①预测区域,②标注信息,③重要性分数,最后得到被定义的实体------目标检测网络。

  但是这篇专利有\textbf{被驳回的风险},因为根据其各个权利组成的方案中,更加偏向于\textbf{智力活动的处理},而且,并没有应用到具体领域,虽说使用了多主权来保证后续的授权的可能性,但是有可能会因为过于宽泛,而很容易被找到\textbf{对比文件}(对比文件指的是可以用来引证当前技术方案并非首创,或者是常规技术,或者是当前领域已熟知技术,或者是不足的文本文件,发表过的论文也可以被引用为对比文件),\textbf{从而被驳回}。

  此处可以学习的地方偏少,但是因为这个是在专利审核改革后的例子,所以可以学习的点有两个:

\begin{itemize}
% \tightlist
\item
  文书化的语言,不管是标注信息,还是目标检测网络,以及独权三中的第一确定模块以及第二确定模块。
\item
  专利布局值得学习。
\end{itemize}

  类似于这个的专利还有公开号为\textbf{CN109977956A}的\textbf{一种图像处理方法、装置、电子设备以及存储介质},书写方式也类似,但是在领域上更加靠近\textbf{文本处理以及图像识别}。所以这个在一定程度上反而更加容易下授权。

\subsection{19.3.3
申请国内发明专利提示}\label{ux7533ux8bf7ux56fdux5185ux53d1ux660eux4e13ux5229ux63d0ux793a}

  下边给出一个精心调整过的申请该类专利的模板。\href{./img/1.doc}{模板下载}(注意,不是这个模板也可以,但是要符合顺序)

  注意,摘要附图以及说明书附图可以无图(仅指发明),还有一些具体的格式要求,可以自己去查询专利申请的要求,这里给出最常见的几个注意点:

\begin{itemize}
% \tightlist
\item
  说明书摘要不能多于300字
\item
  发明内容不能有其特征在于,但是权利要求书需要有其特征在于
\item
  有益效果仅限于独权的范围内,不能超出独权一范围之外
\item
  说明书除发明内容中以外不能有``所述''以及''如权利要求x所述的x``以及''其特征在于``
\item
  一旦附图,必须准确地描述该附图的标题
\item
  说明书附图中,除了步骤图以外,其它的图中必须带有箭头标注或者弯曲线标注
\end{itemize}

\section{19.4
如何保护自己的代码?}\label{ux5982ux4f55ux4fddux62a4ux81eaux5df1ux7684ux4ee3ux7801}

  计算机代码相关的只能申请\textbf{软件著作权},关于软著的保护范围大都限定于实现某个软件/系统的部分代码,举一个例子:

  软件著作权保护的内容其一是指为了得到某种结果而可以以计算机等具有信息处理能力的装置执行的代码化指令序列,或者可以被自动转换成代码化指令序列的符号化指令序列或者符号化语句序列:

  例如下文这种是代码化指令序列(此处用 C 语言的 Hello World 举例):

\begin{Shaded}
\begin{Highlighting}[]
\PreprocessorTok{#include }\ImportTok{<stdio.h>}
\DataTypeTok{int}\NormalTok{ main()}
\NormalTok{\{}
\NormalTok{   printf(}\StringTok{"Hello, World!"}\NormalTok{);}
   \ControlFlowTok{return} \DecValTok{0}\NormalTok{;}
\NormalTok{\}}
\end{Highlighting}
\end{Shaded}

  其它类似于这种的都可以称为代码化指令序列。而符号化语句序列一般指的是汇编语言以及其它一些已经脱离了代码化的符号化语言。但是在我们申请的过程中,需要的是将这个转化为可运行的计算机程序,而且原则上,专利法是\textbf{不保护如此简单的程序代码}的。但是如果你使用的是汇编语言写的上述这段代码,此处的意思就是足够复杂,那这段代码就是属于可被予以考虑的范围。

  而在深度学习中的绝大多数代码都是基于 Python 的,在此例中展示的仅仅是 C
语言例子,并不是指必须要 C 语言,任何语言皆可。

\subsection{19.4.1
已经开源的算法代码怎么保护?}\label{ux5df2ux7ecfux5f00ux6e90ux7684ux7b97ux6cd5ux4ee3ux7801ux600eux4e48ux4fddux62a4}

  需要采取必要的源代码许可证,例如(商业授权许可证) GPLv2.0 或者
GPLv3.0,并且对该代码进行软著申请。

\subsection{19.4.2
论文中的算法实现代码怎么保护?}\label{ux8bbaux6587ux4e2dux7684ux7b97ux6cd5ux5b9eux73b0ux4ee3ux7801ux600eux4e48ux4fddux62a4}

  如果你认为自己的代码具有很强的商业价值,你需要首先对该代码进行软著申请后,然后再考虑以上述的
GPLv2.0
等许可证进行开源(主要是为了保证自己的代码不会被公司用于商业环境中)。

  但是在此需要说一个可能被规避的方式,因为软件著作权的保护仅限于当前语言,如果你真的想要保护该代码的话,可能不仅仅需要申请单语言的代码保护,你还需要对它所折射出来的算法流程进行保护,也就是通过后文提到的软件(算法)专利申请来保护自己的算法流程。

\section{19.5
如何申请软件(算法)专利(发明专利)?}\label{ux5982ux4f55ux7533ux8bf7ux8f6fux4ef6ux7b97ux6cd5ux4e13ux5229ux53d1ux660eux4e13ux5229}

软件专利(发明专利)的申请流程:

\begin{itemize}
% \tightlist
\item
  专利申请
\item
  国知局进行受理回复
\item
  初审(会有补正意见通知书在这期间寄出)
\item
  公布(会有审查意见通知书或者补正意见通知书在这期间寄出)
\item
  实质审查请求(会有审查意见通知书或者补正意见通知书在这期间寄出)
\item
  实质审查(会有审查意见通知书或者补正意见通知书在这期间寄出)
\item
  授权
\end{itemize}

\textbf{你需要做的事情是}:

\begin{itemize}
% \tightlist
\item
  开通自己的专利账户,\href{http://cponline.sipo.gov.cn/}{中国专利电子申请网}中点击注册,成功后获取``电子证书''
\item
  (2019-07-16)如果你的年收入为五万以下,或者你的企业的年收入为一百万以下,可以申请费减备案,可以减免大部分的专利申请费用:\href{http://cpservice.sipo.gov.cn/index.jsp}{专利事务服务系统},注册后,填写添加费减(以及相关资料)后,按照系统提示快递证明文件。通过后可以开始下一步。
\item
  安装 \href{http://cponline.sipo.gov.cn/tooldown/index.jhtml}{CPC
  程序},要同时下载CPC安装文件以及离线升级程序(需要win7+word07)
\item
  选择申请专利,选择发明专利后,填写所有相关的信息后(注意,此处可以使用上述的文件的分节的pdf版)(例如说明书摘要、摘要附图-图n、权利要求书等的pdf文件),保存全部后可以选择签名,使用在第一项中获取的电子证书后,可以点击发送按钮。
\item
  在发送后一到两天,就会被受理,然后点击``接收按钮'',就可以查收你在专利受理通知书还有专利费减通知书(如果审核通过的话)了
\item
  然后通过登录\href{http://cponline.sipo.gov.cn/}{中国专利电子申请网},根据你上述过程中获取的专利受理通知书中提示的专利申请号,就可以对应地进行缴费了
\item
  时刻注意相关的审查意见通知书以及补正意见通知书的下发,随时进行相应的答复以及驳通,同样地也是使用
  CPC 软件进行处理后打包发送。
\end{itemize}

\section{19.6
如何申请软件著作权?}\label{ux5982ux4f55ux7533ux8bf7ux8f6fux4ef6ux8457ux4f5cux6743}

软件著作权的申请流程为:

\begin{itemize}
% \tightlist
\item
  申报流程著作权申请;
\item
  查看著作权登记注意事项;
\item
  整理、提交材料至软件协会;
\item
  版权中心反馈受理通知书(网上公布,此时可申请软件产品登记);
\item
  版权中心发放著作权证书(网上公布);
\item
  企业(个人)到软件协会领取证书。
\end{itemize}

\textbf{你需要做的事情是:}

\begin{itemize}
% \tightlist
\item
  进入\href{http://www.ccopyright.com.cn/}{中国版权保护中心}仔细研读软件\href{http://www.ccopyright.com.cn/index.php?optionid=1033}{登记指南},然后按照其中的要求撰写自己的申请书;准备好资料,所需文件可以在\href{http://www.ccopyright.com.cn/index.php?optionid=1080}{此处}进行查看。
\end{itemize}

\textbf{你需要小心的问题是:}

\begin{itemize}
% \tightlist
\item
  登记申请表的打印必须保证打印出来的申请表是每页一面。
\item
  用户手册中的截图需要保证自己的名称之类的与申请的软件著作权的名称是一致的。
\item
  申请名称必须以品牌+\textbf{软件}主体+(``软件''、``系统''或者``平台'')结尾,不能以
  APP
  结尾或者其它任何形式;如果是手机软件,必须要加(ios版)或者(android版)
\item
  代码必须是对应自己提供的代码的,例如 Java 代码不能说是 JSF,C++
  不能说是 C。
\item
  代码最后一面必须是结束的形式。
\end{itemize}

\textbf{一些细节}:

\begin{itemize}
% \tightlist
\item
  版本号必须是 V或者v + 数字的形式,可以是 v1.0 ,但是更推荐 v1.0.0
\item
  硬件环境(二者择其一):
\item
  CPU、内存、硬盘大小等
\item
  手机型号
\item
  代码行数以实际为准。【3】
\end{itemize}

\subsection{19.6.1
如何为机器学习的代码申请软著?}\label{ux5982ux4f55ux4e3aux673aux5668ux5b66ux4e60ux7684ux4ee3ux7801ux7533ux8bf7ux8f6fux8457}

  注意,实际上你代码写就那天起,你就获取了自己的代码的著作权,但是如果你因为自己的代码被拷贝,所以需要去告别人的时候,你需要首先进行软著的申请,然后才能够对违法行为进行反击。

  如果你是被别人告了,那你预先进行的软著申请就可以对自己产生一定的保护效果。

  那么,你如何才能为自己写的代码,而非成型``软件''进行申请呢?其实这个问题很简单,\textbf{你只需要给你的代码赋予``系统''名称结尾即可,例如:Bon图像识别系统。}可能你实际上连界面都没,但是也可以对你的代码进行一定程度的保护。

  但是要注意,你的代码行数是5000行内的话,那你可能需要将自己的全部代码都彻底公开(但是如果是真的要申请的话,可以对自己代码的关键部分进行涂黑,不显示相关的代码),其实原则上,\textbf{我不推荐仅仅是实现了一个算法的代码进行软著的申请。}

\section{19.7
如何申请非中国范围内的算法专利?}\label{ux5982ux4f55ux7533ux8bf7ux975eux4e2dux56fdux8303ux56f4ux5185ux7684ux7b97ux6cd5ux4e13ux5229}

  假设上述流程中,你是已经到达了准备进行专利递交的申请了,那你在这一步需要同时选择
PCT 申请,否则的话,如果在递交了中国范围内的发明专利申请后,才进行 PCT
的申请,那就会在进入其它国家的时候因为三性问题(新颖性、创造性和工业实用性)【4】,导致被驳回。递交了
PCT
申请以后,可以更加方便地进行外国的专利的申请,你可以要求你所要求申请的国家给予你在中国范围内申请的专利同样的保护权限。

  而如果你希望自己的专利可以获得美国的专利的授权,或者欧盟专利的授权的话,那你就需要去请当地的代理所\textbf{(不再能自己进行申请)}对你的专利进行申请,接下来的流程与国内也是相似的。

  特别地,\textbf{当你想要申请其它国家的专利时,你的专利文本必须要翻译成当地的语言}。

\subsection{19.7.1
美国专利申请流程}\label{ux7f8eux56fdux4e13ux5229ux7533ux8bf7ux6d41ux7a0b}

\paragraph{19.7.1.1
巴黎公约途径}\label{ux5df4ux9eceux516cux7ea6ux9014ux5f84}

  在中国申请后,在第一在线专利申请日(即优先权日------你申请当天算起)起\textbf{12}个月内向美国知识产权局提出专利申请,可以享受优先权待遇(此处优先权指的是假设有人比你迟一个月申请,那就算他先你一步提出了优先权申请,你的优先权日还是比他早)。

\paragraph{19.7.1.2 PCT 途径}\label{pct-ux9014ux5f84}

  在进行 PCT 申请的时候,CPC 程序已经预设了直接 PCT
申请的途径,特别地,PCT
是在中国在先申请的申请日(优先权日)起12个月内向中国国家知识产权局提出PCT国际申请,在自优先权日起30个月内向美国知识产权局提出进入申请;而实际上,一般决定申请美国专利,就需要在初始就开始
PCT 的申请,并且在确定的 30 个月以内向美国知识产权局进行申请即可。

\subsection{19.7.2 能不能在美国申请算法专利,然后走 PCT
途径回国?}\label{ux80fdux4e0dux80fdux5728ux7f8eux56fdux7533ux8bf7ux7b97ux6cd5ux4e13ux5229ux7136ux540eux8d70-pct-ux9014ux5f84ux56deux56fd}

  不行,因为算法就算走 PCT
回国,然后还是会因为\textbf{专利法第二十五条的规定}而\textbf{容易}被驳回。

\subsection{19.7.3
申请国外的专利是否能自己进行?}\label{ux7533ux8bf7ux56fdux5916ux7684ux4e13ux5229ux662fux5426ux80fdux81eaux5df1ux8fdbux884c}

  并不行,目前不管任何国家或者地区中,申请当地的专利都必须要有当地国籍的代理人参与,并且必须按当地的申请流程进行准备。

\section{19.8
如何在写专利前进行适当检索?}\label{ux5982ux4f55ux5728ux5199ux4e13ux5229ux524dux8fdbux884cux9002ux5f53ux68c0ux7d22}

  目前业内使用较多的搜索引擎为佰腾以及soopat,但是谷歌专利也可以胜任大部分情况下的检索。

  在专利行业中优选地检索方式有三种:

  ①技术领域检索,在软件(算法)领域中,常用的技术领域检索为\textbf{计算机装置}以及\textbf{计算机设备}。

  ②关键词检索,在软件(算法)领域中,常用的关键词检索为\textbf{装置}、\textbf{系统}以及\textbf{方法}。

  ③说明书检索,此处是直接输入自己的算法名,因为在说明书中一般会直接概括算法名。

% \section{19.9 参考文献}\label{ux53c2ux8003ux6587ux732e}

% 【1】http://fmyzl.com/html/2013/shenchashijianyuyanjiu\_0703/140.html

% 【2】http://www.gov.cn/flfg/2008-12/28/content\_1189755.htm

% 【3】https://www.zhihu.com/question/20850680

% 【4】http://www.cnipr.com/xy/swzs/sqzc/201707/t20170718\_214375.html

% \ldots{}.

% 未完待续!
